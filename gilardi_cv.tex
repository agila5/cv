%% Copyright 2006-2013 Xavier Danaux (xdanaux@gmail.com).
%
% This work may be distributed and/or modified under the
% conditions of the LaTeX Project Public License version 1.3c,
% available at http://www.latex-project.org/lppl/.

% class and style
\documentclass[11pt,a4paper,sans]{moderncv}
\moderncvstyle{banking}
\moderncvcolor{blue}

% character encoding
\usepackage[T1]{fontenc}
\usepackage[utf8]{inputenc}
\usepackage[english]{babel}

% adjust the page margins
\usepackage[scale=0.85]{geometry}
\nopagenumbers{}

% The following latex code is used to extend the width of the social links column and merge all of them in the same line. 
% Code taken from: https://tex.stackexchange.com/questions/478130/how-do-i-get-all-of-the-social-links-onto-one-line?rq=1
\usepackage{etoolbox}
\patchcmd{\makehead}% <cmd>
{0.8\textwidth}% <search>
{\textwidth}% <replace>
{}{}% <success><failure>

% personal data
\name{Andrea}{Gilardi}
\title{Curriculum Vitae} 
\date{\today}
\address{}{Olginate (LC), Italy}{}{}
\phone[mobile]{+39 3478111403}   
\email{a.gilardi5@campus.unimib.it}
\social[github]{agila5}

% I'm not sure how to extend the \social option for SO and so on. This is an alternative approach taken from https://tex.stackexchange.com/questions/187962/extending-moderncv-social-command
\extrainfo{\href{https://stackoverflow.com/users/10280411/agila}{\faStackOverflow{} agila}}

% \homepage{https://github.com/agila5} TODO
% \photo[64pt][0.4pt]{picture} % optional, remove / comment the line if not wanted; '64pt' is the height the picture must be resized to, 0.4pt is the thickness of the frame around it (put it to 0pt for no frame) and 'picture' is the name of the picture file
% \quote{Some quote} TODO

% to show numerical labels in the bibliography (default is to show no labels); only useful if you make citations in your resume
% \makeatletter
% \renewcommand*{\bibliographyitemlabel}{\@biblabel{\arabic{enumiv}}}
% \makeatother

\begin{document}
	\makecvtitle
	\vspace{-0.5cm}
	\section{Summary}
	\cvitem{}{I am a PhD Student in Statistics and Mathematical Finance at the University of Milan - Bicocca. My main research interests are spatial and spatiotemporal models on road networks.}
	
	\section{Education}
	\cventry{NOV.17--Present}{PhD Student in Statistics and Mathematical Finance}{University of Milan - Bicocca}{}{}{}
	\cventry{SET.15--OTT.17}{Master Degree in Statistics}{University of Milan - Bicocca}{}{}{Final mark: 110/110 e lode (Magna cum laude)} 
	\cventry{SET.12--SET.15}{Bachelor Degree in Statistics}{Università of Milan - Bicocca}{}{}{Final mark: 110/110 e lode (Magna cum laude)}
	
	\section{Teaching}
	\cvitemwithcomment{Teaching Assistant}{Introduction to Statistical Inference, University of Milan - Bicocca}{SEP.20--FEB.21}
	\cvitemwithcomment{Teaching Assistant}{R for Road Safety. \underline{\href{https://www.racfoundation.org/introduction-to-r-for-road-safety}{Program}} and \underline{\href{https://cran.r-project.org/web/packages/stats19/vignettes/stats19-training.html}{Materials}}.}{OCT.19--OCT.19}
	\cvitemwithcomment{Teaching Assistant}{Introduction to Statistical Inference, University of Milan - Bicocca}{SEP.19--FEB.20}
	\cvitemwithcomment{Tutor}{Introduction to R, University of Milan - Bicocca. \underline{\href{https://github.com/agila5/bbetween_R}{Materials}}.}{MAR.19--APR.19}
	\cvitemwithcomment{Tutor}{Introduction to \textit{tidyverse} and \textit{plumber}, University of Turin - Master in Data Science}{OCT.18--OCT.18}
	\cvitemwithcomment{Teaching Assistant}{Introduction to Statistical Inference, University of Milan - Bicocca}{SEP.18--FEB.19}
	\cvitemwithcomment{Teaching Assistant}{Statistica Economica M., University of Milan - Bicocca}{APR.18--JUL.18}
	\cvitemwithcomment{Teaching Assistant}{Introduction to Statistical Inference, University of Milan - Bicocca}{SEP.17--SEP.18}
	\cvitemwithcomment{Teaching Assistant}{Informatics, University of Milan - Bicocca}{MAY.17--SEP.17}

	\section{Working Experience}
	\cventry{OCT.20--NOV.20}{Consultant}{Catholic University of the Sacred Heart, Milan}{Milan (Italy)}{}{I worked on a project named \textit{Black Box and Big Data per la stima e la previsione dei sinistri} under the supervision of Prof. Diego Zappa.}
	\cventry{JUN.20--OCT.20}{Collaboration with the Department of Economics, Management and Statistics}{University of Milan - Bicocca}{Milan (Italy)}{}{Prepare and organize the MSc admission tests.}
	\cventry{APR.18--OCT.18}{Consultant}{GR \& Partner S.A.S}{Milan (Italy)}{}{I developed a User Interface using R and Shiny.}
	
	\section{Master Degree Thesis}
	\cvitem{Title}{\emph{Previsione spaziotemporale per gli interventi in emergenza delle ambulanze nel comune di Milano}}
	\cvitem{Supervisor}{Prof. Riccardo Borgoni.}
	\cvitem{Summary}{We developed a spatiotemporal model to predict the distribution of the emeregency medical interventions in Milan.}
	
	\section{Visiting Periods}
	\cvlistitem{I spent four months in Castellón de la Plana (ES), from February 2020 to June 2020, working with Prof. Jorge Mateu on statistical models for events on road networks.}
	\cvlistitem{I stayed in Leeds (UK) for three months, from September 2019 to November 2019, working with Prof. Robin Lovelace on R packages and classes for road networks. \underline{\href{https://github.com/agila5/leeds_seminar}{Final presentation}.}}
	
	\section{Publications}
	\begin{enumerate}
		\item \textbf{Gilardi, A.}, Mateu, J., Borgoni, R., Lovelace, R., Multivariate hierarchical analysis of car crashes data considering a spatial network lattice. Under review. \underline{\href{https://arxiv.org/abs/2011.12595}{Preprint}}.  
		\item \textbf{Gilardi, A.}, Lovelace, R., Padgham, M. Street Networks in R. Under review. \underline{\href{https://osf.io/78yub}{Preprint}}.
		\item Borgoni, R., \textbf{Gilardi, A.}, \& Zappa, D. (2020). Assessing the Risk of Car Crashes in Road Networks. Social Indicators Research, 1-19.
		\item \textbf{Gilardi, A.}, Borgoni, R., \& Zappa, D. (2020). Assessing the Risk of Car Accidents in Road Networks. In NESPUTT 2019-New Economic \& Statistical Perspectives on Urban \& Territorial Themes.
		\item \textbf{Gilardi, A.}, Borgoni, R., \& Zappa, D. (2019). Spatial Logistic Regression for Events Lying on a Network: Car Crashes in Milan. In Smart Statistics for smart applications (pp. 1165-1170). Pearson Italia.
		\item \textbf{Gilardi, A.}, Borgoni, R., Pagliosa, A., \& Bonora, R. (2018). Spatiotemporal Prevision for Emergency Medical System Events in Milan. In Scientific meeting of the Italian Statistical Society.
	\end{enumerate}

	\section{Conferences}
	\cvlistitem{OpenGeoHub 2020, Wageningen (Netherlands)}
	\cvlistitem{e-RUM 2020: European R User Meeting. Online event.}
	\cvlistitem{Nesputt 2019: New Economic \& Statistical Perspective, Milan (Italy).}
	\cvlistitem{UseR! 2019 Meeting, 9-12 July 2019, Tolouse (France).}
	\cvlistitem{50th Meeting of the Italian Statistical Society, 18-21 June 2019, Milan (Italy).}
	\cvlistitem{Statalk, March 29th, 2019, University of Padova (Italy).}
	\cvlistitem{49th Meeting of the Italian Statistical Society, 20-22 June 2018, Palermo (Italy).}
	\cvlistitem{Statalk, November 17th, 2017, University of Padova (Italy).}
	
	\section{Organized Conferences}
	% \cvlistitem{plotKML}
	\cvlistitem{I organized an online Webinar and Hackathon related to the R package \href{https://github.com/luukvdmeer/sfnetworks}{\underline{\texttt{sfnetworks}}} as official satellite events of the e-Rum 2020 Conference. \underline{\href{https://github.com/sfnetworks/sfnetworks-webinar}{Materials}}, \underline{\href{https://sfnetworks.github.io/sfnetworks-webinar/slides\#1}{slides}} and \underline{\href{https://www.youtube.com/watch?v=ZXTNXsvKYwo}{recordings}}.}
	\cvlistitem{I was a member of the Local Organizing Committee of \underline{\href{https://www.nesputt2019.unimib.it/}{Nesputt 2019: New Economic \& Statistical Perspective}}, Milan (Italy).}
	
	\section{Grants}
	\cvlistitem{\underline{\href{https://www.r-consortium.org/projects/awarded-projects/2019-group-2\#Tidy+spatial+networks+in+R}{Tidy Spatial Networks in R}} funded by the \underline{\href{https://www.r-consortium.org/}{R Consortium}}. 9000\$.}

	\section{Awards}
	\cvlistitem{Scolarship for the OpenGeoHub 2020 - Summer School. Project: \textit{plotKML functionality to support sf and stars objects}. \underline{\href{https://www.youtube.com/watch?v=n3SjuJIEnKY&list=PLXUoTpMa_9s0Ea--KTV1OEvgxg-AMEOGv&index=42}{Final presentation}}.}
	\cvlistitem{\underline{\href{https://www.sis-statistica.it/}{SIS}} award for the best poster at the SIS-2019 conference.}
	\cvlistitem{\underline{\href{https://youngsis.github.io/}{Young-SIS}} award for the best poster at the SIS-2018 conference.}
	
	\section{Software}
	\cvitem{}{I am a proficient R user, with a particular interest in the \texttt{r-spatial} world and the tidyverse environment. I am the author of two R packages, namely \underline{\href{https://github.com/ITSLeeds/osmextract}{\texttt{osmextract}}} (currently \underline{\href{https://github.com/ropensci/software-review/issues/395}{under review}} by rOpenSci foundation) and \underline{\href{https://github.com/luukvdmeer/sfnetworks}{\texttt{sfnetworks}}}. 
	I contributed to several R packages like \texttt{ggplot2}, \texttt{recipes}, \texttt{sf}, \texttt{stplanr} and \texttt{osmdata}. 
	I attended the \textit{Tidyverse Developer Day} in Tolouse during the UseR! 2019 conference. 
	}
	\cvitem{}{I have basic knowledge of SAS, Python and STATA, and \LaTeX.}	
		
	\section{Badges}
	\cvitem{SAS}{\underline{\href{https://www.youracclaim.com/badges/9b93b851-47a9-4ff9-91f7-433cbfa0fdb2/public_url}{SAS Certified Base Programmer for SAS 9}}}
	
	\cvitem{SAS}{\underline{\href{https://www.youracclaim.com/badges/d6743078-d280-4bdb-816c-08ca4d736010/public_url}{SAS Certified Predictive Modeler Using SAS Enterprise Miner 7}}}
	
	
	\section{Additional Information}
	\cvitem{Associazioni}{Volunteer at Volontari del Soccorso di Calolziocorte from July 2016}
	
	\section{Notes}
	Autorizzo il trattamento dei miei dati personali ai sensi del Decreto Legislativo 30 giugno 2003, n. 196 "Codice in materia di protezione dei dati personali".
	
	%\section{References}
	%\begin{cvcolumns}
	%	\cvcolumn{Category 1}{\begin{itemize}\item Person 1\item Person 2\item Person 3\end{itemize}}
	%	\cvcolumn{Category 2}{Amongst others:\begin{itemize}\item Person 1, and\item Person 2\end{itemize}(more upon request)}
	%	\cvcolumn[0.5]{All the rest \& some more}{\textit{That} person, and \textbf{those} also (all available upon request).}
	%\end{cvcolumns}
\end{document}


%% end of file `template.tex'.
