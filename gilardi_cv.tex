%% start of file `template.tex'.
%% Copyright 2006-2013 Xavier Danaux (xdanaux@gmail.com).
%
% This work may be distributed and/or modified under the
% conditions of the LaTeX Project Public License version 1.3c,
% available at http://www.latex-project.org/lppl/.

\documentclass[11pt,a4paper,sans]{moderncv}
\moderncvstyle{banking}
\moderncvcolor{blue}

% character encoding
\usepackage[utf8]{inputenc}

% adjust the page margins
\usepackage[scale=0.75]{geometry}
\setlength{\hintscolumnwidth}{2.85cm}
\nopagenumbers{}

% I want to extend the width and get all social links on the same line. 
% Code taken from: https://tex.stackexchange.com/questions/478130/how-do-i-get-all-of-the-social-links-onto-one-line?rq=1
\usepackage{etoolbox}
\patchcmd{\makehead}% <cmd>
{0.8\textwidth}% <search>
{\textwidth}% <replace>
{}{}% <success><failure>

% personal data
\name{Andrea}{Gilardi}
\title{Curriculum Vitae} 
\date{\today}
\address{}{Olginate (LC), Italy}{}{}
\phone[mobile]{+39 3478111403}   
\email{a.gilardi5@campus.unimib.it}
\social[github]{agila5}

% I'm not sure how to extend the \social option for SO and so on. This is an alternative approach taken from https://tex.stackexchange.com/questions/187962/extending-moderncv-social-command
\extrainfo{\href{https://stackoverflow.com/users/10280411/agila}{\faStackOverflow{} agila}}

% \homepage{https://github.com/agila5} 
% \photo[64pt][0.4pt]{picture}                       % optional, remove / comment the line if not wanted; '64pt' is the height the picture must be resized to, 0.4pt is the thickness of the frame around it (put it to 0pt for no frame) and 'picture' is the name of the picture file
% \quote{Some quote}

% to show numerical labels in the bibliography (default is to show no labels); only useful if you make citations in your resume
\makeatletter
\renewcommand*{\bibliographyitemlabel}{\@biblabel{\arabic{enumiv}}}
\makeatother

\begin{document}
	\makecvtitle
	\section{Summary}
	\cvitem{}{I am a PhD student in Statistics and Mathematical Finance at the University of Milan - Bicocca. My main research interests are spatial and spatiotemporal models on road networks.}
	
	\section{Education}
	\cventry{NOV.17--Present}{PhD Student in Statistics and Mathematical Finance}{University of Milan - Bicocca}{}{}{}
	\cventry{SET.15--OTT.17}{Master Degree in Statistics}{University of Milan - Bicocca}{}{}{Final mark: 110/110 e lode (Magna cum laude)} 
	\cventry{SET.12--SET.15}{Bachelor Degree in Statistics}{Università of Milan - Bicocca}{}{}{Final mark: 110/110 e lode (Magna cum laude)}
	
	\section{Working Experience}
	\cventry{OCT.19--OCT.19}{Teaching assistant}{University of Leeds}{}{}{I helped Dr. Robin Lovelace teaching a course on using R for Road Safety. \underline{\href{https://cran.r-project.org/web/packages/stats19/vignettes/stats19-training.html}{Materials}}}
	\cventry{SEP.19--FEB.20}{Teaching assistant}{University of Milan - Bicocca}{}{}{Course: Statistica II - Introduction to Statistical Inference.}
	\cventry{MAR.19--APR.19}{Tutor}{University of Milan - Bicocca}{}{}{I taught a short introductory course (20h) related to the R software. \underline{\href{https://github.com/agila5/bbetween_R}{Materials}}.}
	\cventry{OCT.18--OCT.18}{Tutor}{University of Turin - Master in Data Science}{}{}{I taught a short introductory lesson (4h) related to the R packages \textit{tidyverse} and \textit{plumber}.}
	\cventry{SEP.18--FEB.19}{Teaching assistant}{University of Milan - Bicocca}{}{}{Course: Statistica II - Introduction to Statistical Inference.}
	\cventry{APR.18--OCT.18}{Consultant}{GR \& Partner S.A.S}{Milan (Italy)}{}{I developed a User Interface using R and Shiny.}
	\cventry{APR.18--JUL.18}{Teaching assistant}{University of Milan - Bicocca}{}{}{Course: Statistica Economica M.}
	\cventry{SEP.17--SEP.18}{Teaching assistant}{University of Milan - Bicocca}{}{}{Course: Statistica II - Introduction to Statistical Inference.}
	\cventry{MAY.17--SEP.17}{Teaching assistant}{University of Milan - Bicocca}{}{}{Course: Informatica - Informatics.}
	
	\section{Master Degree Thesis}
	\cvitem{Title}{\emph{Previsione spaziotemporale per gli interventi in emergenza delle ambulanze nel comune di Milano}}
	\cvitem{Supervisor}{Prof. Riccardo Borgoni}
	\cvitem{Summary}{We developed a spatiotemporal model to predict the distribution of the emeregency medical interventions in Milan.}
	
	\section{Visiting Periods}
	\cvlistitem{I spent four months in Castellón de la Plana (ES), from February 2020 to June 2020, working with Prof. Jorge Mateu on statistical models for events on road networks.}
	\cvlistitem{I stayed in Leeds (UK) for three months, from September 2019 to November 2019, working with Prof. Robin Lovelace on R packages and classes for road networks. \underline{\href{https://github.com/agila5/leeds_seminar}{Final presentation}.}}
	
	\section{Publications}
	\begin{enumerate}
		\item \textbf{Gilardi, A.}, Lovelace, R., Padgham, M. Street Networks in R. Under review. 
		\item Borgoni, R., \textbf{Gilardi, A.}, \& Zappa, D. (2020). Assessing the Risk of Car Crashes in Road Networks. Social Indicators Research, 1-19.
		\item \textbf{Gilardi, A.}, Borgoni, R., \& Zappa, D. (2020). Assessing the Risk of Car Accidents in Road Networks. In NESPUTT 2019-New Economic \& Statistical Perspectives on Urban \& Territorial Themes.
		\item \textbf{Andrea, G.}, Riccardo, B., \& Zappa, D. (2019). Spatial Logistic Regression for Events Lying on a Network: Car Crashes in Milan. In Smart Statistics for smart applications (pp. 1165-1170). Pearson Italia.
		\item \textbf{Gilardi, A}., Borgoni, R., Pagliosa, A., \& Bonora, R. (2018). Spatiotemporal Prevision for Emergency Medical System Events in Milan. In Scientific meeting of the Italian Statistical Society.
	\end{enumerate}

	\section{Conferences}
	% \cvlistitem{OpenGeoHub 2020, Wageningen (Netherlands)}
	\cvlistitem{e-RUM 2020: European R User Meeting. Online event.}
	\cvlistitem{Nesputt 2019: New Economic \& Statistical Perspective, Milan (Italy).}
	\cvlistitem{UseR! 2019 Meeting, 9-12 July 2019, Tolouse (France).}
	\cvlistitem{50th Meeting of the Italian Statistical Society, 18-21 June 2019, Milan (Italy).}
	\cvlistitem{Statalk, March 29th, 2019, University of Padova (Italy).}
	\cvlistitem{49th Meeting of the Italian Statistical Society, 20-22 June 2018, Palermo (Italy).}
	\cvlistitem{Statalk, November 17th, 2017, University of Padova (Italy).}
	
	\section{Organized Conferences}
	% \cvlistitem{plotKML}
	\cvlistitem{I organized an online Webinar and Hackathon on the R package \href{https://github.com/luukvdmeer/sfnetworks}{\underline{\texttt{sfnetworks}}} as official satellite events of the e-Rum 2020 Conference.}
	\cvlistitem{I was a member of the Local Organizing Committee of \href{https://www.nesputt2019.unimib.it/}{\textcolor{blue}{Nesputt 2019: New Economic \& Statistical Perspective}}, Milan (Italy).}

	\section{Awards}
	\cvlistitem{SIS award for the best poster at the SIS-2019 conference.}
	\cvlistitem{Young-SIS award for the best poster at the SIS-2018 conference.}
	
	\section{Software}
	\cvitem{}{I am a proficient R user, with a particular interest in the \texttt{r-spatial} packages and the tidyverse environment. I attended the \textit{Tidyverse Developer Day} in Tolouse during the UseR! 2019 conference. I am the author of one R package (not on CRAN at the moment) called \underline{\href{https://github.com/ITSLeeds/osmextract}{\textit{osmextract}}}. I contributed to several R packages like \textit{ggplot2}, \textit{recipes}, \textit{sf}, \textit{stplanr} and \textit{osmdata}. I have basic knowledge of SAS, Python and STATA. I can use \LaTeX~ for writing scientific documents.}	
		
	\section{Badges}
	\cvitem{SAS}{\underline{\href{https://www.youracclaim.com/badges/9b93b851-47a9-4ff9-91f7-433cbfa0fdb2/public_url}{SAS Certified Base Programmer for SAS 9}}}
	
	\cvitem{SAS}{\underline{\href{https://www.youracclaim.com/badges/d6743078-d280-4bdb-816c-08ca4d736010/public_url}{SAS Certified Predictive Modeler Using SAS Enterprise Miner 7}}}
	
	
	\section{Additional Information}
	\cvitem{Associazioni}{Volunteer at Volontari del Soccorso di Calolziocorte from July 2016}
	
	\section{Notes}
	Autorizzo il trattamento dei miei dati personali ai sensi del Decreto Legislativo 30 giugno 2003, n. 196 "Codice in materia di protezione dei dati personali".
	
	%\section{References}
	%\begin{cvcolumns}
	%	\cvcolumn{Category 1}{\begin{itemize}\item Person 1\item Person 2\item Person 3\end{itemize}}
	%	\cvcolumn{Category 2}{Amongst others:\begin{itemize}\item Person 1, and\item Person 2\end{itemize}(more upon request)}
	%	\cvcolumn[0.5]{All the rest \& some more}{\textit{That} person, and \textbf{those} also (all available upon request).}
	%\end{cvcolumns}
\end{document}


%% end of file `template.tex'.
