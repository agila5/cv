% !TeX program = xelatex

%% Copyright 2006-2013 Xavier Danaux (xdanaux@gmail.com).
%
% This work may be distributed and/or modified under the
% conditions of the LaTeX Project Public License version 1.3c,
% available at http://www.latex-project.org/lppl/.

\PassOptionsToPackage{unicode, colorlinks, urlcolor=black}{hyperref}

% class and style
\documentclass[11pt,a4paper,sans]{moderncv}
\moderncvstyle{banking}
\moderncvcolor{black}

% Color 
\definecolor{myblue}{rgb}{0.22,0.45,0.70}

% character encoding
\usepackage{fontspec}
\usepackage{xurl}
% \setmainfont{Times New Roman}

% adjust the page margins
\usepackage[scale=0.875]{geometry}
\nopagenumbers{}

% The following latex code is used to extend the width of the social links column and merge all of them in the same line. 
% Code taken from: https://tex.stackexchange.com/questions/478130/how-do-i-get-all-of-the-social-links-onto-one-line?rq=1
\usepackage{etoolbox}
\patchcmd{\makehead}% <cmd>
{0.75\textwidth}% <search>
{\textwidth}% <replace>
{}{}% <success><failure>

% personal data
\name{Andrea}{Gilardi}
\title{Curriculum Vitae} 
\date{\today}
\address{}{Milan, Italy}{}{}
% \phone[mobile]{+39 3478111403}   
\email{andrea.gilardi@polimi.it}
\social[github]{agila5}
\social[orcid]{0000-0002-9424-7439}

% I'm not sure how to extend the \social option for SO and so on. This is an alternative approach taken from https://tex.stackexchange.com/questions/187962/extending-moderncv-social-command
\extrainfo{\href{https://stackoverflow.com/users/10280411/agila}{\faStackOverflow{} agila}}

\homepage{agila5.github.io}
% \photo[64pt][0.4pt]{picture} % optional, remove / comment the line if not wanted; '64pt' is the height the picture must be resized to, 0.4pt is the thickness of the frame around it (put it to 0pt for no frame) and 'picture' is the name of the picture file
% \quote{Some quote} TODO

% to show numerical labels in the bibliography (default is to show no labels); only useful if you make citations in your resume
% \makeatletter
% \renewcommand*{\bibliographyitemlabel}{\@biblabel{\arabic{enumiv}}}
% \makeatother

% \usepackage{titlesec}

\begin{document}
  \maketitle
  \vspace{-0.75cm}
  
  \hypersetup{urlcolor=blue}
  
  \section{Summary}
  \cvitem{}{
  I am an Assistant Professor (Ricercatore SECS-S/01, Legge 240/10 tipo A) at the \href{https://www.mate.polimi.it/?lg=en}{Department of Mathematics} of \href{https://www.polimi.it/}{Politecnico di Milano}. I am a member of the \href{https://mox.polimi.it/}{MOX} laboratory of mathematical modelling and scientific computing. My main research interests focus on spatial and spatiotemporal models for events on linear networks.
  }

  \section{Current Position}
  \cventry{JUN. 23 - Today}{Assistant Professor (RTDA)}{Politecnico di Milano}{Milan, Italy}{}{I am working on a project named \emph{GRINS: Growing Resilient, INclusive and Sustainable} which is part of the Next Generation EU program (National Recovery and Resilience Plan).}

  \section{Education}
  \cventry{NOV.17--APR.21}{PhD in Statistics and Mathematical Finance}{University of Milano - Bicocca}{}{}{}
  \cventry{SET.15--OTT.17}{Master Degree in Statistics \normalfont{(LM-82)}}{University of Milano - Bicocca}{}{}{Final mark: 110/110 e lode (Magna cum laude)} 
  \cventry{SET.12--SET.15}{Bachelor Degree in Statistics \normalfont{(L-41)}}{Università of Milano - Bicocca}{}{}{Final mark: 110/110 e lode (Magna cum laude)}
	
  \section{Teaching}
  \subsection{Lecturer (Docente)}
  \cvlistitem{\emph{R for Data Science}, University of Milano - Bicocca, PhD in Economics, Management and Statistics, Curriculum Statistics, 12 hours, a.y. 2022-2023. \href{https://github.com/agila5/R4DS-PhD-Unimib}{Materials}.}
  \cvlistitem{\emph{Livellamento al Software R} (Introduction to the R Software), University of Milano - Bicocca, Master degree in Statistics and Economics, 14 hours, a.y. 2022-2023. \href{https://github.com/agila5/livellamento-R}{Materials}.}
  
  \subsection{Teaching Assistant (Esercitatore)}
   \cvlistitem{\emph{Statistica II} (Statistics II), University of Milano - Bicocca, Bachelor degree in Statistical and Economic Sciences, 20 hours, aa.yy. 2017-2018; 2022-2023; 2023-2024.}
  
  \subsection{Tutor}
  \cvlistitem{\emph{Statistica II} (Statistics II), University of Milano - Bicocca, Bachelor degree in Statistical and Economic Sciences, 40 hours per year, aa. yy. 2018-2019; 2019-2020; 2020-2021; 2021-2022.}
  \cvlistitem{\href{https://bnews.unimib.it/blog/lab-data-challenges-la-sfida-che-proietta-gli-studenti-nel-mondo-del-lavoro}{\emph{Lab Data Challenges}}, University of Milano - Bicocca, Master degree in Statistics and Economics, 20 hours, a.y. 2020-2021.}
  \cvlistitem{\emph{R for Road Safety}, RAC Foundation, 12 hours, Oct. 2019, \href{https://www.racfoundation.org/introduction-to-r-for-road-safety}{Program} and \href{https://cran.r-project.org/web/packages/stats19/vignettes/stats19-training.html}{Materials}.}
  \cvlistitem{\emph{Introduction to R}, University of Milano - Bicocca, Bbetween 2019 MULTIMEDIA, 20 hours, a.y. 2018-2019, \href{https://github.com/agila5/bbetween_R}{Materials}.}
  \cvlistitem{\emph{Introduction to tidyverse and plumber}, University of Turin - Master in Data Science, 4 hours, Oct. 2018.}
  \cvlistitem{\emph{Statistica Economica M} (Economic Statistics),  University of Milano - Bicocca, Master degree in  Statistics and Economics, 20 hours, a.y. 2017-2018.}
  \cvlistitem{\emph{Informatica} (Informatics),  University of Milano - Bicocca, Bachelor degree in Statistical and Economic Sciences, 20 hours, a.y. 2016-2017.}

  \section{Working Experience}
  \cventry{JAN. 21 - JUN. 23}{Research fellow (Assegnista di Ricerca)}{University of Milano - Bicocca}{Milan, Italy}{}{I worked on a project entitled "Spatio-temporal models for spatial network data" under the supervision of Prof. \href{https://www.unimib.it/riccardo-borgoni}{Riccardo Borgoni}.}
  \cventry{OCT.20--NOV.20}{Tutor}{Catholic University of the Sacred Heart}{Milan (Italy)}{}{I worked on a project named \textit{Black Box and Big Data per la stima e la previsione dei sinistri} under the supervision of Prof. \href{https://docenti.unicatt.it/ppd2/it/docenti/04485/diego-zappa/profilo}{Diego Zappa}.}
  \cventry{JUN.20--OCT.20 \& JUN.21 -- OCT.21 \& JUN.22 -- OCT.22}{Tutor}{University of Milano - Bicocca}{Milan (Italy)}{}{I prepared and organized the admission tests for the master degree in Statistics and Economics, Department of Economics, Management and Statistics.}
  \cventry{APR.18--OCT.18}{Consultant}{GR \& Partner S.A.S}{Milan (Italy)}{}{I developed a User Interface using R and Shiny.}

  \section{PhD Thesis}
  \cvitem{Title}{\emph{Statistical Models and Data Structures for Spatial Data on Road Networks}}
  \cvitem{Supervisor}{Prof. Riccardo Borgoni and Prof. Jorge Mateu}
  \cvitem{Summary}{The thesis introduces data structures, vital pre-processing steps, and statistical models to analyse spatial data lying on road networks using point-pattern and network-lattice approaches.}
  \section{Master Degree Thesis}
  \cvitem{Title}{\emph{Previsione spaziotemporale per gli interventi in emergenza delle ambulanze nel comune di Milano}}
  \cvitem{Supervisor}{Prof. Riccardo Borgoni}
  \cvitem{Summary}{We developed a spatiotemporal model to predict the distribution of the emergency medical interventions in Milan.}
	
  \section{Visiting Periods}
  \cvlistitem{I spent seven months in Castellón de la Plana (ES), from February 2020 to June 2020 and May 2021 to July 2021, working with Prof. \href{https://www3.uji.es/~mateu/}{Jorge Mateu} on statistical models for events on road networks.}
  \cvlistitem{I stayed in Leeds (UK) for three months, from September 2019 to November 2019, working with Prof. \href{https://environment.leeds.ac.uk/transport/staff/953/dr-robin-lovelace}{Robin Lovelace} on R packages and classes for road networks. \href{https://github.com/agila5/leeds_seminar}{Final presentation}.}
	
  \section{Publications}
  \begin{enumerate}
  \item \textbf{Gilardi, A.}, Lovelace, R., Padgham, M., Data structures and methods for reproducible street network analysis: overview and implementations in R. Under review. \href{https://osf.io/78yub}{Preprint};
  \item \textbf{Gilardi, A.}, Borgoni, R., Mateu, J., A non-separable first-order spatio-temporal intensity for events on linear networks: an application to ambulance interventions. (2023+) Accepted for publication in \textit{Annals of Applied Statistics}. \href{https://arxiv.org/abs/2106.00457}{Preprint};
  \item \textbf{Gilardi, A.}, Borgoni, R., The impact of traffic flow and road signs on road accidents: an approach based on spatiotemporal point pattern analysis on linear networks. Accepted for presentation at the SIS 2023 conference. 
  \item \textbf{Gilardi, A.}, Borgoni, R., Presicce, L., Mateu, J. (2023) Measurement Error Models for Spatial Network Lattice Data: Analysis of car crashes in Leeds. \textit{Journal of the Royal Statistical Society: Series A (Statistics in Society)} Volume 186, Issue 3, July 2023, Pages 313–334, \url{https://doi.org/10.1093/jrsssa/qnad057};
  \item \textbf{Gilardi, A.}, Borgoni, R., Mateu, J. (2022). Geographically weighted regression for spatial
  network data: an application to traffic volumes
  estimation. In 51th Scientific Meeting of the Italian Statistical Society (pp. 1504 - 1509). Torino:Pearson Italia. ISBN: 9788891932310, Caserta (IT). \href{https://it.pearson.com/content/dam/region-core/italy/pearson-italy/pdf/Docenti/Universit%C3%A0/Sis-2022-4c-low.pdf}{Link}. 
  \item \textbf{Gilardi, A.}, Borgoni, R., Mateu, J. (2022). Spatial statistical calibration on linear networks: an application to the analysis of traffic volumes. In Proceedings of the 10th International Workshop on Spatio-Temporal Modelling, Lleida (Spain), 1-3 June 2022. ISBN 978-84-9144-364-3. \url{http://doi.org/10.21001/METMA\_X}. 
  \item Borgoni, R., \textbf{Gilardi, A.}, Zappa, D. (2022). Optimal Subgrids from Spatial Monitoring Networks. In: (a cura di): Lombardo R; Camminatiello I; Simonacci V, IES 2022 Innovation \& Society 5.0: Statistical and Economic Methodologies for Quality Assessment - Book of Short Papers. p. 148-152, Sesto San Giovanni:Professional Knowledge Empowerment, ISBN: 978-88-94593-35-8, Capua, 2022.  \href{https://drive.google.com/file/d/1HEZG50lecMyx_NDRLi2Y7d0papiX9ZWO/view}{Link}.
  \item \textbf{Gilardi, A.}, Mateu, J., Borgoni, R., Lovelace, R. (2022) Multivariate hierarchical analysis of car crashes data considering a spatial network lattice. \textit{Journal of the Royal Statistical Society: Series A (Statistics in Society)} Volume 185, Issue 3, July 2022, Pages 1150–1177, \url{http://doi.org/10.1111/rssa.12823};
  \item \textbf{Gilardi, A.}, Borgoni, R., Presicce, L., Mateu, J. (2021) Measurement error models on spatial network lattices: car crashes in Leeds. In 13th Scientific Meeting of the Classification and Data Analysis Group, Firenze, September 9-11, 2021. \url{https://doi.org/10.36253/978-88-5518-340-6}; 
  \item \textbf{Gilardi, A.}, Borgoni, R., Mateu, J. (2021). A spatio-temporal model for events on road networks: an application to ambulance interventions in Milan. In 50th Scientific Meeting of the Italian Statistical Society (pp. 702 - 707). Torino:Pearson Italia. ISBN: 9788891927361, Pisa - Virtual. \href{https://it.pearson.com/content/dam/region-core/italy/pearson-italy/pdf/Docenti/Universit%C3%A0/pearson-sis-book-2021-parte-1.pdf}{Link}; 
  \item Borgoni, R., \textbf{Gilardi, A.}, Zappa, D. (2020). Assessing the Risk of Car Crashes in Road Networks. Social Indicators Research, 1-19. \url{https://doi.org/10.1007/s11205-020-02295-x};
  \item Borgoni, R., \textbf{Gilardi, A.}, Zappa, D. (2020). Spatial modelling of driver crash risk using georeferenced data. In NESPUTT 2019 - New Economic \& Statistical Perspectives on Urban \& Territorial Themes. (NESPUTT 2019). E-book. vol. libreriauniversitaria, ISBN: 9788833592091, University of Milan - Bicocca, 2019. \href{https://www.libreriauniversitaria.it/ebook/9788833592091/autore-riccardo-borgoni/new-economic-statistical-perspectives-on-urban-and-territorial-themes-nesputt-2019-e-book.htm}{Link};
  \item \textbf{Gilardi, A.}, Borgoni, R., Zappa, D. (2020). Assessing the Risk of Car Accidents in Road Networks. In NESPUTT 2019 - New Economic \& Statistical Perspectives on Urban \& Territorial Themes. (NESPUTT 2019). E-book. vol. libreriauniversitaria, ISBN: 9788833592091, University of Milan - Bicocca, 2019.  \href{https://www.libreriauniversitaria.it/ebook/9788833592091/autore-riccardo-borgoni/new-economic-statistical-perspectives-on-urban-and-territorial-themes-nesputt-2019-e-book.htm}{Link};
  \item \textbf{Gilardi, A.}, Borgoni, R., Zappa, D. (2019). Spatial Logistic Regression for Events Lying on a Network: Car Crashes in Milan. In: Book of Short Papers SIS2019. (pp. 1165-1170), Pearson, ISBN: 9788891915108, Milano, 2019.   \href{https://it.pearson.com/content/dam/region-core/italy/pearson-italy/pdf/Dirigenti%20e%20istituzioni/ISTITUZIONI-HE-PDF-sis2019_V4.pdf}{Link};
  \item \textbf{Gilardi, A.}, Borgoni, R., Pagliosa, A., Bonora, R. (2018).  Spatiotemporal Prevision for Emergency Medical System Events in Milan. In: Book of short Papers SIS 2018 (pp. 1697-1702). ISBN: 9788891910233, Palermo, Italia, 2018. \href{https://it.pearson.com/content/dam/region-core/italy/pearson-italy/pdf/Dirigenti%20e%20istituzioni/ISTITUZIONI%20-%20HE%20-%20PDF%20-%20SIS%20V4.pdf}{Link}. 
  \end{enumerate}

  \section{Summer Schools and Workshops}
  \cvlistitem{R Project Sprint 2023, University of Warwick (UK), August 30 - September 1, 2023. \href{https://contributor.r-project.org/r-project-sprint-2023/}{Webpage.}}
  \cvlistitem{Contributing to R, Online workshop held during useR! 2021. % \href{https://www.youtube.com/watch?v=CZmldTOdlRM}{Recordings}.
  }
  \cvlistitem{Spatial Statistics for huge datasets and best practices, Online workshop held during useR! 2021. % \href{https://www.youtube.com/watch?v=gQEM-cpLrmg}{Recordings.}
  }
  \cvlistitem{Tidyverse developer day, Tolouse (FR), July 2019. \href{https://www.tidyverse.org/blog/2019/04/tidyverse-dev-day-at-user-2019/}{Announcement.}}
  \cvlistitem{Robust Statistics : Foundations and Recent Developments, University of Milano-Bicocca, March 2019.}
  \cvlistitem{CLADAG, Summer School on Clustering and Classification, University of Catania - May 2018.}
  
  \section{Conferences}
  
  \subsection{Invited}
  \cvlistitem{2023 IMS International Conference on Statistics and Data Science (ICSDS), December 18-21, 2023, Lisbon, Portugal}
  \cvlistitem{15th International Conference of the ERCIM WG on Computational and Methodological Statistics (CMStatistics 17-19 December 2022) - King’s College London.}
  \cvlistitem{Summer Institute in Computational Social Science (SICSS-Covenant), 22-29 June 2022, Covenant University, Ogun State, Nigeria. Online Event.}
  \cvlistitem{OpenGeoHub Summer School 2021, 1-3 September 2021, Online Event.}
  
  \subsection{Contributed}
  \cvlistitem{Free and Open Source Software for Geospatial (FOSS4G), 22-28 August 2022, Florence (Italy).}
  \cvlistitem{51th meeting of the Italian Statistical Society, 22-24 June 2022, Caserta (Italy).}
  \cvlistitem{METMA X, 10th International Workshop on Spatio-Temporal Modelling, 1-3 June 2022, Lleid (Spain)}
  \cvlistitem{CLADAG 2021, 13th Scientific Meeting Classification and Data Analysis Group, 9-11 September 2021, Online.}
  \cvlistitem{UseR! 2021 Meeting, 5-9 July 2021, Online.}
  \cvlistitem{50th meeting of the Italian Statistical Society, 21-25 June 2021, Online.}
  \cvlistitem{OpenGeoHub 2020, 16-23 August 2020, Wageningen (Netherlands).}
  \cvlistitem{Nesputt 2019: New Economic \& Statistical Perspective, 21-22 November 2019, Milan (Italy).}
  \cvlistitem{Meeting of the Italian Statistical Society, 18-21 June 2019, Milan (Italy).}
  \cvlistitem{49th Meeting of the Italian Statistical Society, 20-22 June 2018, Palermo (Italy).}
  
  \subsection{Attended}
  \cvlistitem{GRASPA 2023, 10-11 July 2023, Palermo (Italy).}
  \cvlistitem{Workshop Young Environmental Statistician (YES), February 3rd, 2023, University of Bologna (Italy).}
  \cvlistitem{e-Rum 2020: European R User Meeting, 17-20 June 2020, Online event.}
  \cvlistitem{UseR! 2019 Meeting, 9-12 July 2019, Tolouse (France).}
  \cvlistitem{Statalk, March 29th, 2019, University of Bologna (Italy).}
  \cvlistitem{Statalk, November 17th, 2017, University of Padova (Italy).}
	
  \section{Organized Conferences}
  \cvlistitem{I organized an online Webinar and Hackathon related to the R package \href{https://github.com/luukvdmeer/sfnetworks}{\texttt{sfnetworks}} as official satellite event of the e-Rum 2020 Conference. \href{https://github.com/sfnetworks/sfnetworks-webinar}{Materials}, \href{https://sfnetworks.github.io/sfnetworks-webinar/slides\#1}{slides} and \href{https://www.youtube.com/watch?v=ZXTNXsvKYwo}{recordings}.}
  \cvlistitem{I was a member of the Local Organizing Committee of \href{https://www.nesputt2019.unimib.it/}{Nesputt 2019: New Economic \& Statistical Perspective} conference, Milan (Italy).}
  
  \section{Grants}
  \cvlistitem{\href{https://www.r-consortium.org/projects/awarded-projects/2019-group-2\#Tidy+spatial+networks+in+R}{Tidy Spatial Networks in R} funded by the \href{https://www.r-consortium.org/}{R Consortium}. 9000\$. Joint project with Lucas van der Meer, \href{https://scholar.google.com/citations?user=vDqPwpUAAAAJ&hl=en&oi=sra}{Lorena Abad}, and \href{https://scholar.google.com/citations?user=xDJHVCAAAAAJ&hl=en&oi=ao}{Robin Lovelace.}}
  
  \section{Awards}
  \cvlistitem{Travel award to attend the R Project Sprint 2023 event. 300£.}
  \cvlistitem{Scolarship for the OpenGeoHub 2020 - Summer School. Project: \textit{plotKML functionality to support sf and stars objects}. 800€. \href{https://www.youtube.com/watch?v=n3SjuJIEnKY&list=PLXUoTpMa_9s0Ea--KTV1OEvgxg-AMEOGv&index=42}{Final presentation}.}
  \cvlistitem{\href{https://www.sis-statistica.it/}{SIS} award for the best poster at the SIS-2019 conference.}
  \cvlistitem{\href{https://youngsis.github.io/}{Young-SIS} award for the best poster at the SIS-2018 conference.}
  
  \hypersetup{urlcolor=black}
  
  \section{Community activities}
  \cvitem{}{\textbf{Referee service}}
  \cvitem{}{\href{https://www.springer.com/journal/10182}{AStA Advances in Statistical Analysis};  
  \href{https://ropensci.org/}{rOpenSci};
  \href{https://www.nature.com/srep/}{Scientific Reports}; \href{https://www.sciencedirect.com/journal/spatial-statistics}{Spatial Statistics}; \href{https://www.springer.com/journal/10260}{Statistical Methods \& Applications}; \href{https://www.tandfonline.com/journals/utch20}{Technometrics}.}
  \vspace*{0.15cm}
  \cvitem{}{\textbf{Affiliations}}
  \cvitem{}{\href{https://imstat.org/}{Institute of Mathematical Statistics (IMS)}; \href{https://www.isi-web.org/}{International Statistical Institute (ISI)};   \href{https://ropensci.org/}{rOpenSci}; \href{https://www.sis-statistica.it/}{Società Italiana di Statistica (SIS)}.}
  \vspace*{0.15cm}
  \cvitem{}{\textbf{Projects and activities with public engagement}}
  \cvlistitem{Member of the group working on Progetto di Terza Missione entitled \textit{Analisi della dislocazione spaziale e spaziotemporale degli interventi extraospedalieri e definizione di criteri per monitorare e dimensionare la flotta dei mezzi di soccorso}, University of Milano - Bicocca, 2023.}

  \hypersetup{urlcolor=blue}
  
  \section{Software}
  \cvitem{}{I am a proficient R user, with a particular interest in the \texttt{r-spatial} world and the tidyverse environment. I am the author of two R packages, namely \href{https://github.com/ITSLeeds/osmextract}{\texttt{osmextract}} (\href{https://github.com/ropensci/software-review/issues/395}{reviewed} by rOpenSci foundation and now on \href{https://cran.r-project.org/package=osmextract}{CRAN}) and \href{https://github.com/luukvdmeer/sfnetworks}{\texttt{sfnetworks}} (also on \href{https://cran.r-project.org/package=sfnetworks}{CRAN}). The first package focuses on the management and downloading of Open Street Map data extracts obtained from several providers (like \href{https://download.geofabrik.de/}{Geofabrik}), while the second package provides a tidy approach to spatial network analysis.}
  
  \cvitem{}{I contributed to several R packages like \texttt{ggplot2}, \texttt{recipes}, \texttt{sf}, \texttt{stplanr} and \texttt{osmdata}. 
  I attended the \href{https://www.tidyverse.org/blog/2019/04/tidyverse-dev-day-at-user-2019/}{\textit{Tidyverse Developer Day}} in Tolouse during the UseR! 2019 conference and I plan to attend the \href{https://contributor.r-project.org/r-project-sprint-2023/}{R Project Sprint 2023} event at the University of Warwick (UK). 
  }
  \cvitem{}{I can write scientific documents and presentations using \LaTeX~and I have a basic knowledge of SQLite, GDAL, POSTGIS, SAS, Python, and STATA.}
  
  \section{Badges}
  \cvitem{SAS}{\href{https://www.youracclaim.com/badges/9b93b851-47a9-4ff9-91f7-433cbfa0fdb2/public_url}{SAS Certified Base Programmer for SAS 9}}
  
  \cvitem{SAS}{\href{https://www.youracclaim.com/badges/d6743078-d280-4bdb-816c-08ca4d736010/public_url}{SAS Certified Predictive Modeler Using SAS Enterprise Miner 7}}
  
  \section{Additional Information}
  \cvitem{Associazioni}{Volunteer at Volontari del Soccorso di Calolziocorte from July 2016.}
  
  \section{Notes}
  Autorizzo la pubblicazione ai sensi del D.Lgs. n. 33/2013 "Riordino della disciplina riguardante gli obblighi di pubblicità, trasparenza e diffusione di informazioni da parte delle pubbliche amministrazioni" e acconsento all’utilizzo delle informazioni ivi contenute ai sensi del D.L. n. 196/2003 "Codice in materia di protezione dei dati personali". \\
  
  Le dichiarazioni rese nel presente curriculum sono da ritenersi rilasciate ai sensi degli artt. 46 e 47 del D.P.R. 445/2000.
  
  %\section{References}
  %\begin{cvcolumns}
  %	\cvcolumn{Category 1}{\begin{itemize}\item Person 1\item Person 2\item Person 3\end{itemize}}
  %	\cvcolumn{Category 2}{Amongst others:\begin{itemize}\item Person 1, and\item Person 2\end{itemize}(more upon request)}
  %	\cvcolumn[0.5]{All the rest \& some more}{\textit{That} person, and \textbf{those} also (all available upon request).}
  %\end{cvcolumns}
\end{document}


%% end of file `template.tex'.
